\section{Environment and actions}
In contrast to other approaches on a 2D lattice, I will instead 
model the environment using a 3D FCC lattice, which has the added benefit of
being computationally trivial to model. I make use of the
matrix representation:
\begin{equation}
    \begin{split}
        & \mathbf{e} = \begin{bmatrix}
            0 & 0.5 & 0.5 \\
            0.5 & 0 & 0.5 \\
            0.5 & 0.5 & 0
        \end{bmatrix} \\
    \end{split}
\end{equation}
Any point in space is represented as a linear combination of the
column space of the bravaise lattice, within a discrete action setting,
this can be viewed as a choice among $3^3$ possible actions $ a = \begin{bmatrix}
    x \\ y \\ z
\end{bmatrix}$ s.t $(x,y,z) \in \{-1,0,1 \}$.
The chains of amino acids are initialised as a straight line in space, and each
time-step each residue moves to another position $x_t$ by selecting a movement vector
$a_{\mathbf{j}}$ to be applied to it's current position. The transition function
is characterised as:
\begin{equation}
    T(x_t, a_{\mathbf{j}}) =\mathbf{e} \cdot a_{\mathbf{j}}
\end{equation}
At any point in time, I also consider the joint state $m_{x_t}$ as described by \cite{Mguni2018}
to model the densities of residues in 3D space.
\section{Revising reward structures}
Instead of using the standard rewards associated with the $HP$ model, 
I will instead divide each of the residues into their respective categories
under the $hHPNX$ model with the signs of values inverted, each residue acts as their own agent with their own
reward function determined according to the column space in table (2.3). In
addition to these rewards, the agents are further penalized with with a reward
of -10 for occupying the same space as another residue to discourage overlapping residues:
\begin{table}[!htb]
    \caption{Reward structure}
    \begin{center}
        \caption{}
        \begin{tabular}{|c || c | c | c | c | c|}
            \hline
             & h & H & P & N & X \\
            \hline
            h & -2 & 4 & 0 & 0 & 0 \\
            \hline
            H & 4 & 3 & 0 & 0 & 0 \\
            \hline
            P & 0 & 0 & -1 & 1 & 0\\
            \hline
            N & 0 & 0 & 1 & -1 & 0\\
            \hline
            X & 0 & 0 & 0 & 0 & 0\\ 
            \hline
            $x' = x$ & -10 & -10 & -10 & -10 & -10\\
            \hline
        \end{tabular}
    \end{center}
\end{table}\\
In addition to these rewards, the agents are also weighted proportionally to 
the spatial congestion as described by \cite{Mguni2018} as follows:
\begin{equation}
    R(m_{x_t},a') = \psi R_{hHPNX} + (1 - \psi)  \frac{\exp (-(x_t - \mu)^\top \Sigma^{-1} (x_t - \mu))}{2 \pi \sqrt{\vert \Sigma \vert}(1 + m_{x_t})^\alpha}
\end{equation}
Where $\alpha < 0$ encourages the occupation of dense areas and $0 < \psi < 1$ is a hyperparameter
that weights the agent's intrinsic reward against the global reward, with higher values
of $\psi$ favouring intrinsic reward more than global reward. This serves two purposes,
it offsets the sparse rewards of the neutral group $X$ and it serves to provide
a richer training signal in the proteins initial denatured state, guiding it to the
hydrophobic collapse, both rewards are complementary and provide different benefits at 
different stages of training, $\psi$ is set initially low $\backsim 0.2$ and annealed up to $\backsim 0.9$.
This heterogenous reward structure is afforded by the gurantees provided by \cite{Sriram2020},
which ensures convergence with lower bounds within the mean-field scheme when considering 
multiple types of agents.                
\section{Mean field multi-type spatial congestion games}
As suggested previously, each residue is modelled as an individual agent belonging to one of the
categories as listed in the hHPNX model. The reward structure of each type is heterogenous and so can be considered
to be members of disjoint types. I model the inter-domain cooperativity by positively rewarding
the agents for the occupation of lattice sites with:
\begin{itemize}
    \item Favourable contacts with neighbouring residues 
    \item Dense occupation of other residues
\end{itemize}
Rewards for the occupation of dense areas of space encourages cooperativity amongst all
agents and acts as a guide to the most compact states. The reward for favourable contacts
also filters the landscape of possible conformations leaving only those that maximise
the number of hydrophic contacts while being optimally compact, this mirrors the
work of \cite{Yang} regarding the two tiers of interactions. The reward for
dense packing is initially high, driving the fast interactions due rapidly
accumulating rewards as the density increases. As the value of $\psi$ is annealed,
the specific contacts among residues becomes more important as they rearrange themselves
while in a compact form, this resembles the tier 2 of slower interactions that guide the 
proteins to the bottom of the \emph{Gibbs energy funnel}. The use of the \emph{hHPNX} scheme
on a 3D FCC lattice also reduces the number of degenerative results as \cite{Hoque} shows.\\

Each agent's Q function is updated in accordance with the multi-type paradigm which was shown to be provably convergent:
\begin{equation}
    Q^j_{t+1}(s,a^j,a^{-j}, a^{-j}_1, \hdots , a^{-j}_M)= (1-\alpha)Q^j_t(s,a^j,a^{-j}, a^{-j}_1, \hdots , a^{-j}_M) +\alpha[R^j(m_{x_t}, a^j)+ \gamma v^j_t(s')]
\end{equation}
Where $a^j$ represents the action taken by the central agent, $a^{-j}$ and $a^{-j}_m$ represents
the empirical distribution of all neighbours and neighbours of type $m$ respectively.
\section{Risk sensitive agents}
As \cite{Xueguang2018} show, distributional reinforcement learning proves to be effective
in multi-agent settings; an agent is better able to \emph{distribute blame} proportionally
among the actions of neighbouring agents. For instance, for a given transition $s\rightarrow s'$,
if we consider the joint action taken by all agents $\mathbf{a}$, then a central agent $j$ may determine
that the action taken by a specific agent $\mathbf{a}_k$ was the most influential factor in the
transition to specific state $s'$; this is also known as the \emph{credit assignment problem}. In their work, they use quantile networks to determine a state-specific learning rate $\alpha$
that changes based on the interaction with other agents, becoming more or less exploratory
as the situation deems.\\

I considered this approach, aswell as an alternative approach that shaped the the reward accord to the
predicted distribution of of returns for the actions selected by neighbours. However I instead argue that


\subsection{Reward Shaping with Quantile Estimates}
Reward shaping is an approach to optimal credit assignment in environments with sparse rewards.

